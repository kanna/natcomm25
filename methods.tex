
\section{Methods}
% max 3000 words

\subsection{Overall Approach}

Our methodology is based on the implementation of a complete data cycle to enhance model predictions in a coastal ocean domain through adaptive sampling and data assimilation. The approach consists of three fundamental steps, executed iteratively on a daily basis:

\begin{enumerate}
    \item \textbf{Model Forecast and Uncertainty Projection}:  
    A numerical ocean model provides a daily one-step forecast $\hat{\theta}(k+1, x, y)$ of a target oceanic variable $\theta$, along with an associated uncertainty field $\sigma_{\hat{\theta}}(k+1, x, y)$, where $k$ represents the current day and $(x, y)$ denote the geographical coordinates. These outputs are organized into discrete spatial maps, $M_{\hat{\theta}}(x, y)$ and $M_{\sigma_{\hat{\theta}}}(x, y)$, representing, respectively, the predicted state and its uncertainty over a predefined grid covering the study area.
    
    \item \textbf{Adaptive Sampling Planning}:  
    Using the uncertainty map $M_{\sigma_{\hat{\theta}}}(x, y)$ as input, an adaptive sampling algorithm determines the set of trajectories for a fleet of $N$ Autonomous Underwater Vehicles (AUVs) for the next operational cycle. The goal is to maximize the accumulated uncertainty sampled along the vehicle paths, while satisfying vehicle-specific constraints such as maximum endurance, operational limits, and navigation feasibility. The planned trajectories are transmitted to the vehicles for execution.

    \item \textbf{Data Collection and Assimilation}:  
    Throughout the operational period, each AUV collects pointwise measurements of the target variable $\theta$ along its assigned path. After the mission is completed, the collected measurements are assimilated into the numerical model using an appropriate data assimilation scheme. This updated model state serves as the new initial condition for the next forecasting cycle, closing the loop.
\end{enumerate}

\subsection{Available Models}
\subsection{Adaptive Sampling Algorithm}
\subsection{Available robotic systems and assets}

% The adaptive sampling problem is formulated as an optimization task aiming to maximize the expected information gain from the observations. Specifically, given the predicted uncertainty map and operational constraints (e.g., time, energy budget), the algorithm plans vehicle routes that prioritize areas of higher model uncertainty. The uncertainty along each candidate path is evaluated, and path planning strategies — such as greedy heuristics or combinatorial optimization techniques — are employed to allocate waypoints efficiently among the available AUVs.

\subsection{Operational Constraints and Practical Considerations}

% The practical implementation of the data cycle in a real-world marine environment introduces several constraints:

% \begin{itemize}
%     \item \textbf{Communication Limitations}:  
%     Low-bandwidth and intermittent communications at sea require that mission planning be sufficiently robust to accommodate long periods of autonomous operation without human intervention.
    
%     \item \textbf{Vehicle Constraints}:  
%     AUV endurance, payload limitations, navigation precision, and deployment risks all impose restrictions on the feasible operational space and mission duration.

%     \item \textbf{Computational Demands}:  
%     Real-time generation of uncertainty fields, optimization of paths, and assimilation of collected data must be performed within time windows compatible with daily operational cycles, often under constrained computational resources.

%     \item \textbf{Environmental Variability}:  
%     Fast-evolving coastal ocean dynamics can introduce discrepancies between forecasted and actual conditions, necessitating robust planning that accounts for forecast uncertainty and adaptivity.
% \end{itemize}

% This method was deployed and evaluated under the operational framework of the FRESNEL project, providing a real-world demonstration of the data cycle’s feasibility and benefits in complex coastal environments.


% Distinction Between Onboard and Offboard Predictions:
% \begin{itemize}
%     \item Onboard: Statistical prediction
%     \item Offboard: Numerical prediction
% \end{itemize}

% Rationale and Implementation (including a schematic figure):
% \begin{itemize}
%     \item Why this approach?
%     \item How was it executed?
% \end{itemize}

% Adaptive Sampling Strategy:
% \begin{itemize}
%     \item contraints
%     \item Algorithm used for real-time decision-making
%     \item Criteria for data collection optimization
% \end{itemize}

% Statistical Approach:
% \begin{itemize}
%     \item Techniques applied for uncertainty estimation
%     \item Integration with observational data
% \end{itemize}

%  Numerical Model (HOPS - Harvard Ocean Prediction System):
% \begin{itemize}
%     \item Model configuration and setup
%     \item Data assimilation methods
% \end{itemize}


