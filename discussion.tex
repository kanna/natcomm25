\section{Discussion}
\label{sec:disc}


%%%%%%%%%%%% xp2 vs xp3/xp5 results, north and south area %%%%%%%
Although the overall results indicate that the assimilation procedure
generally improves model performance across most scenarios, a more
detailed and vehicle-specific analysis reveals important nuances. Some
of these results highlight aspects that should be carefully considered
in the planning and execution of future oceanographic campaigns of this
nature. These points are discussed following. Figures \ref{fig:rms_c}
and \ref{fig:rms_c} presents the root-mean-square error (RMSE) of
temperature predictions compared with in-situ data from XP2, XP3, and
XP5 under the different assimilation scenarios (C, C4, D, and D4) for
31\textsuperscript{st} October. The results are shown as a function of
depth and vehicle, allowing a direct comparison between simulations with
and without data assimilation. I SUGGEST EXPLAINING THIS BETTER

For XP3 and XP5, a clear and consistent trend is observed: the
assimilation scenarios (C4 and D4) improve the model prediction skill.
The RMSE error decreases across all depths and vehicles, indicating that
the assimilation procedure effectively constrains the local dynamics
represented in the geostatistical model. Although small differences are
present between C4 and D4, they are not statistically significant,
confirming that the assimilation results are robust with respect to
initial configuration differences.

The situation for XP2 is radically different. In this case, the
assimilation scenarios (C4 and D4) lead to higher prediction errors
compared with the non-assimilation runs (C and D). This
counterintuitive result stems from an important spatial mismatch: on
31\textsuperscript{st} October, XP2 operated north of the \naz Canyon,
whereas the assimilated data from the previous days
(29\textsuperscript{th} and 30\textsuperscript{th} October) were
collected in its southern sector. The \naz Canyon marks a transition
zone between distinct oceanographic regimes, where circulation
patterns are strongly influenced by the canyon’s complex topography
and the interaction between shelf and slope processes. Inside the \naz
Canyon, residual currents are generally aligned along the canyon axis
as a result of strong topographical control and this alignment extends
well above the canyon edges (~150 m depth), implying a substantial
disturbance of the predominant north–south circulation parallel to the
general trend of the shelf and slope
\cite{tyler2009europe,relvas2007physical,guerreiro2014influence}.
\rmcomment{we can talk here about the clustering/ allocation by areas
  of different dyNAmics, problems with covergence of the error, etc -
  future studies YES}


%%%% SUBgrid variability - the importance of using autonomous systems %%%%%
Another topic of discussion is the subgrid variability of AUV
temperature observations within the model grid. The analysis of
temperature variability within the model grid provides additional
insight into the subgrid-scale structure of the observed field and the
representativeness of AUV-based measurements. Figure \ref{fig:iqd_3D}
depicts the interquartile distance (IQD) of temperature values
recorded by the AUVs within the three-dimensional discretization of
the statistical model. Each grid cell represents the maximum spatial
resolution available to the model, while the IQR quantifies the local
variability of in situ measurements acquired as the vehicles navigated
through each cell.

This analysis demonstrates that AUVs frequently capture significant
variability at scales smaller than the model resolution, revealing the
existence of fine-scale gradients, in some locations greater than
1$^{\circ}$C, and processes that are unresolved by the model. Although
the statistical model provides smoothed, grid-averaged predictions of
ocean properties, AUV measurements highlight fluctuations that occur
within individual cells, indicating the presence of relevant
subgrid-scale dynamics. These small-scale variations can contribute to
local model–data discrepancies and should be taken into account when
interpreting or assimilating observations.

Across all missions, the AUVs sampled each grid cell between a few and
several hundred times, with median values ranging from 9 to 14
measurements per cell and mean values between 19 and 26. These
statistics indicate a relatively uniform and dense sampling pattern,
sufficient to characterize the local variability of the water column
at the subgrid level. The maximum number of samples per grid cell
reached several thousand (up to 8757 on 31 October for XP5), though
these extreme values are most probably biased by vehicle behavior near
the surface, where the AUVs temporarily remained stationary to
communicate and acquire GPS positioning. As such, these maxima likely
reflect oversampling at specific locations rather than representative
variability across the full domain.

\begin{figure}
    \centering
    \includegraphics[width=1\linewidth]{iqd_xp5_30oct.png}
    \caption{Enter Caption}
    \label{fig:placeholder}
\end{figure}

Despite these localized biases, the overall sampling density achieved
by the AUVs provides a detailed depiction of fine-scale ocean
variability. The IQD distributions reveal higher variability near the
thermocline, where vertical gradients are strongest, while in some
instances increased variability could also appear near the surface,
potentially corresponding to small-scale frontal zones or localized
mixing events. This analysis highlights that the AUVs, despite having
slightly higher sensor uncertainty than traditional ship-based CTD
systems, can capture dynamic subgrid processes that are otherwise
unresolved in model grids.

Importantly, this subgrid variability is not merely observational
noise but a meaningful indicator of local uncertainty and
environmental heterogeneity. Incorporating this information—both the
variability (IQD) and the spatial sampling density—into data
assimilation frameworks could enhance the model’s ability to represent
uncertainty at the subgrid scale and improve the realism of short-term
forecasts.



 %%%%%%%%%%%%%
The challenging ocean and meteorological conditions faced as part of
this experiment precluded cycles of week-long operations, as
planned. Nevertheless, we were able to operate for 7 days during the
3-week long deployment. We demonstrated the overall integrative
overall approach to close modeling-sampling-assimilation-tasking cycle
with the goal of improving the skill of oceanographic models by
leveraging observations from AUVs, traditional methods such as
ship-based measurements and opportunistic measurements using low-cost
sensors, and by exploring synergies between dedicated surveys and
opportunistic observations. ELIMINATE SHIP BASED MEASUREMENTS

Full integration of command and control strategies comprising modeling,
assimilation and adaptive sampling algorithms was demonstrated with the
help of the mature software toolchain enabling 24/x operations with
minimal operator’s support. The deployments were performed to evaluate
and test the overall approach in an incremental fashion. The last days
of the deployment demonstrated several iterations of the closed
modeling-sampling-assimilation-tasking cycle. The accuracy of the
predictions of the geostatistical model increased after several cycles,
while the overall prediction errors decreased. However, these results
still lack statistical significance because of the few consecutive
cycles in which the overall approach was tested. We observed that we
never had more than 3 consecutive days of operation.

The dense grid of AUV observations enables post-experiment evaluation
and testing of the assimilation schemes and of the geostatistical model.
While we are still lacking the desired statistical significance of a
long series of consecutive cycles, these post-experiment activities will
lead to a better understanding of the overall procedures and the
identification of improvements, namely the optimization of the
parameters used for the coordinated integration of the algorithms used
in the modeling-sample-assimilation-tasking cycles. We will further
investigate the selection of representative depths for the application
of the sampling algorithm used to find the horizontal projection of the
AUV paths. THIS HAS NOT BEEN DISCUSSED BEFORE

This field study provided invaluable lessons in operational procedures,
refinement of adaptive sampling and algorithms, and risk minimization to
operate in an area with dense ship traffic and fishing nets. Our AUVs
were occasionally impacted by strong vertical currents that may have
resulted from the impact of internal waves that are common in the area,
namely in stratified regions (which was the case) and that were observed
with the help of remote sensing imagery during the deployments. Finally,
the results achieved with this deployment provided additional insights
and the motivation to further advance the state of the art in refining
the modeling-sampling-assimilation-tasking cycle with the goal of
improving the skill of oceanographic models. Furthermore, dense grids of
sampled oceanographic data have the potential to fuel developments
targeting existing gaps in modeling skill when different levels of
spatial and temporal resolution are considered \cite{Balaji_2022}.


%\begin{figure}[!]
%  \centering
%  \includegraphics[scale=0.3]{fig/rms_bar_vechicle.png}
%  \caption{not the final figure}
%  \label{fig:rms_bar_vechicle}
%\end{figure}

\begin{figure}[!]
  \centering
  \includegraphics[scale=0.3]{fig/iqd_xp5_30.png}
  \caption{Trajectory of the XP5 during the 30 October mission in the
    Nazaré Canyon region. The color shading represents the
    interquartile range (IQD) of temperature values recorded within
    each model grid cell along the vehicle’s path. The grid
    illustrates the horizontal resolution of the statistical model,
    and the results are shown for three representative depth layers:
    near-surface, mid-depth, and around 40 m, capturing distinct
    vertical regimes of variability. Higher IQD values indicate
    regions of higher thermal variability within model grid.}
  \label{fig:iqd_3D}
\end{figure}


