\section{Introduction}
\linenumbers
\label{sec:intro}

The coastal ocean, the marine areas extending from the coastline to
the continental slope, encompassing the continental shelf, constitutes
one of the most dynamic and complex regions of the ocean. These areas
are characterized by high variability, intricate topographies, and
coastline geometries. They are forced by a wide range of physical and
biogeochemical processes which occur across diverse spatial and
temporal scales. They are among the most productive and economically
significant areas of the world’s oceans, benefiting from terrestrial
inputs via river discharges and nutrient renewal driven by upwelling
processes. As a result and in part, coastal areas concentrate a great
proportion of human activities including leisure, fisheries and
offshore aquaculture and renewable energy exploitation. The coastal
ocean mediates interactions between deep-ocean processes and coastal
environment, influencing how large-scale phenomena — such as
climate-driven changes or extreme weather events — affect the coastal
population. It also governs, for example, how anthropogenic influences
originating on land are redistributed and affect marine ecosystems
\cite{greene25}.

Interactions among physical, chemical, and biological processes in the
coastal ocean can amplify or mitigate the impacts of extreme events,
including storms, flooding, and pollution. Furthermore, the coastal
ocean serves as a key interface where large-scale oceanic and
atmospheric changes manifest their consequences for coastal
ecosystems. Accurately forecasting the state of the coastal ocean is
therefore essential not only for safeguarding environmental and
economic interests, but also for enhancing resilience to climate
variability, natural hazards, and anthropogenic stressors.

Ocean models have become essential tools for predicting ocean state
and underpin a wide range of societal applications, from climate
forecasting to pollution monitoring and resource management. However,
these models are inherently imperfect representations of complex
reality. Data assimilation techniques — where observational data are
integrated to update models — play a critical role in correcting or
adjusting model errors and improving their forecasting
capabilities. Yet, the success of data assimilation depends heavily on
the availability, quality, and relevance of observations.

Satellite remote sensing provides extensive spatial coverage but is
generally limited to surface observations and constrained by
atmospheric conditions. In situ observations from ships, moorings, or
drifting platforms, although spatially sparse and logistically
demanding, complement satellite data by providing higher spatial or
temporal resolution and access to subsurface processes, adding a
critical three-dimensional component to the observing system. Recent
technological advances have enabled the use of autonomous underwater
vehicles (AUVs) and other robotic platforms, offering flexible,
mobile, and increasingly autonomous means of sampling ocean properties
with high spatiotemporal resolution and low logistic footprint
\cite{das11b,graham12,das15,fossum18}.

However, even with these new tools, efficiently collecting data for
modeling at optimal spatial and temporal scales remains a complex
challenge. AUVs are constrained not only by endurance and
communication bandwidth, but also by limited transit speed (typically
on the order of 1-2 ms\textsuperscript{-1}), comparable to ambient
current velocities and thus limits synopticity. This limitation is
even more pronounced for gliders, which operate at lower effective
speeds. Operational risks further restrict deployment, including
vessel traffic, fishing activity, strong currents, and complex
bathymetry. In addition, the ocean itself remains highly dynamic, with
key processes often evolving faster than traditional ship-based
sampling strategies can capture. This opens the door to adaptive
sampling strategies, where observation efforts are guided by model
outputs and uncertainty estimates to maximize the relevance of
collected data. Adaptive sampling \cite{BinZhang07,Singh09,fossum19}
represents a fundamental shift: instead of executing pre-planned
missions, autonomous vehicles dynamically prioritize areas for
observation.

Our approach combines model predictions with adaptive sampling with
propelled AUVs. \cite{lermusiaux07} explores such a concept; however
the novelty of our work is in demonstrating the applicability of using
our AUVs that can control their trajectory using onboard computation
and doing so in a dynamic coastal environment. % In doing so
% combines predicted model uncertainties in comparison with prior data
% assimilation process with available observations.

Model forecasts identify where spatial uncertainty is highest or where new
observations are expected to have the greatest impact in reducing
forecast errors, i.e., higher value of information. Vehicles are then
directed to sample these areas, and their data are in turn assimilated
back into the models, as a virtuous cycle. % ; model
% prediction generates a forecast and associated uncertainty field while
% AUV-based adaptive sampling targets areas of maximum predicted
% uncertainty.
Data assimilation from these high uncertainty regions is then
incorporated into the model such that new predictions benefit from
improved in-situ data. Such a data cycle approach promises to
significantly enhance the skill of ocean forecasts, especially in the
complex coastal environment where processes occur over a wide range of
scales.

Our work has three significant results. First it demonstrates the
importance of coupling ocean models with high resolution AUV
sampling. Second, it highlights the importance of targeted adaptive
sampling based on model uncertainty. Third, it demonstrates the impact
of assimilation to predict and close the
\emph{sample-assimilate-predict-direct} loop
(Fig. \ref{fig:block-diag}). Additionally, sampling resolution with
AUVs vastly outweigh the data obtained from traditional ship-based
approaches. Multiple AUVs sampling simultaneously, as we do here, also
makes a dent in spatial as well as temporal coverage of such dynamic
environments at relatively moderate costs. Additionally, AUVs can
provide near real-time data, allowing for their rapid re-tasking and
potentially placement in evolving spatially separated regions.

\begin{figure}[!t]
  \centering
  \includegraphics[scale=0.4]{fig/ensemble-2.jpg}
  \caption{\proj demonstrates the value of integrating ocean models
    with adaptive robotic vehicles in the coastal ocean, to increase
    model skill and prediction within a tight control loop.}
  \label{fig:block-diag}
\end{figure}

This manuscript is organized as follows. Section \ref{sec:intro}
provides an introduction to the problem we are tackling providing a
motivation for the \emph{sample-assimilate-predict-direct}
loop. Section \ref{sec:study-area} provides the environmental context
and the challenge of modeling and operating robotic vehicles in a
dynamic coastal region. Section \ref{sec:methods} highlights the
details of the forecasting, assimilation, sampling and operational
strategies which lie at the core of this paper, while Section
\ref{sec:disc} discusses the results from the experiment and wraps up
with conclusions and future work in Section \ref{sec:conclude}.
