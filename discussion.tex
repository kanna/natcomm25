\section{Discussion}
\label{sec:disc}

Although the overall results indicate that the assimilation procedure
generally improves model skill, a more detailed and vehicle-specific
analysis reveals important nuances. 

Figs. \ref{fig:rms_c} and \ref{fig:rms_d} present the RMSE predictions
compared with in-situ data under the different assimilation scenarios
(\textbf{C}, \textbf{C4}, \textbf{D}, and \textbf{D4}) for
31\textsuperscript{st} October. The results are shown as a function of
depth and vehicle, allowing a direct comparison with and without data
assimilation.  For XP3 and XP5, a clear and consistent trend is
observed: the assimilation scenarios (\textbf{C4} and \textbf{D4})
improves the model prediction skill, with RMSE decreasing across all
depths, indicating that the assimilation procedure effectively
constrains the local dynamics represented in the geostatistical
model. Although small differences are present between \textbf{C4} and
\textbf{D4}, they are not statistically significant and confirm that
assimilation results are robust with respect to initial configuration
differences.

The situation for XP2 was different however. The assimilation
scenarios (\textbf{C4} and \textbf{D4}) lead to higher prediction
errors compared with the non-assimilation runs (\textbf{C} and
\textbf{D}). This counterintuitive result stems from an important
spatial mismatch: on 31\textsuperscript{st} October, XP2 operated
north of the \naz Canyon, whereas the assimilated data from the
previous days (29\textsuperscript{th} and 30\textsuperscript{th}
October) were collected south of it. The \naz Canyon marks a
transition zone between distinct oceanographic regimes, where
circulation patterns are strongly influenced by the canyon’s complex
topography and the interaction between shelf and slope
processes. Inside the canyon, residual currents are generally aligned
along the canyon axis as a result of strong topographical control and
this alignment extends well above the canyon edges ($\sim150$m depth),
implying a substantial disturbance of the predominant north–south
circulation parallel to the general trend of the shelf and slope
\cite{tyler2009europe,relvas2007physical,guerreiro2014influence}.
\rmcomment{we can talk here about the clustering/ allocation by areas
  of different dyNAmics, problems with covergence of the error, etc -
  future studies YES.}


Another highlight of this experiment was in the visibility of
variability within the model grid providing additional insight into
the subgrid-scale structure of the observed
field. Fig. \ref{fig:iqd_3D} depicts the interquartile distance (IQD)
of measurements recorded by the AUVs within the three-dimensional
discretization of the statistical model. Each grid cell represents the
maximum spatial resolution available in the model, while the IQD
quantifies the local variability of in-situ measurements acquired as
the vehicles navigated through each
cell.\kcomment{Fig. \ref{fig:iqd_3D} is very hard to read. Equally,
  might not there be the argument from a reviewer/reader, that to
  overcome this variability, perhaps a higher res. model should be
  used? So it might be worthwhile to make a stronger case for this
  statistical modeling approach in the methods section. See my
  comments there.}

This analysis demonstrates that robotic platform can capture
significant variability at scales smaller than the model resolution,
revealing the existence of fine-scale gradients, in some locations
greater than 1$^{\circ}$C, as also processes that are unresolved by
the model. Although the statistical model provides smoothed,
grid-averaged predictions of ocean properties, AUV measurements here,
highlight fluctuations that occur within individual cells, indicating
the presence of subgrid-scale dynamics. These small-scale variations
can contribute to local model–data discrepancies \kcomment{We need a
  citation or something to buttress this claim.} and should be taken
into account when interpreting or assimilating observations.

Across all missions, the AUVs sampled each grid cell with median
values ranging from 9 to 14 measurements per cell and mean values
\kcomment{Of temperature?}  between 19 and 26. These statistics
indicate a relatively uniform and dense sampling pattern, sufficient
to characterize the local variability of the water column at the
subgrid level. The maximum number of samples per grid cell reached
several thousand (for example up to 8757 on 31\textsuperscript{st}
October for XP5), though these extreme values are most probably biased
by vehicle behavior near the surface, where the AUVs temporarily
remained stationary to communicate and acquire GPS. 

Despite these localized biases, the overall sampling density achieved
by the AUVs provides a detailed depiction of fine-scale ocean
variability. The IQD distributions reveal higher variability near the
thermocline, where vertical gradients are strongest, while in some
instances increased variability could also appear near the surface,
potentially corresponding to small-scale frontal zones or localized
mixing events. This analysis highlights that the AUVs, despite having
slightly higher sensor uncertainty \kcomment{What is the evidence for
  this? When a claim is made, for good or bad, it helps to have it
  substantiated ideally with a citation or some text that walks thru
  the reasoning. By saying this, you're making a claim about the
  resolution of data and immediately deflating your claim. If the
  claim, rightfully I believe, is that robotic sampling is more
  efficient and provides higher resolution insights, then we ought to
  just state it, not make it conditional. In fact unless it is a major
  takeaway from this experiment, we perhaps not mentioned it.} than
traditional ship-based CTD systems, can capture dynamic subgrid
processes that are otherwise unresolved in model grids.

Importantly, this subgrid variability is not merely observational
noise but a meaningful indicator of local uncertainty and
environmental heterogeneity. Incorporating this information—both the
variability (i.e. IQD) and the spatial sampling density—into data
assimilation frameworks could enhance the model’s ability to represent
uncertainty at the subgrid scale and improve the realism of short-term
forecasts. \kcomment{But did we not assimilate \emph{all} the AUV
  measurements at the resolution the vehicles sampled in? If so, why
  the ``could enhance'' as against ``it did enhance''?}

This field study provided invaluable lessons in operational
procedures, refinement of adaptive sampling and algorithms, and risk
minimization to operate in an area with dense ship traffic and fishing
nets. Our AUVs were occasionally impacted by strong vertical currents
that may have resulted from the impact of internal waves that are
common in the area, and that were observed with the help of remote
sensing imagery during the experiment. Finally, the results achieved
with this deployment provided additional insights and the motivation
to further advance the state of the art in refining the
\emph{sample-assimilate-predict-direct} cycle with the goal of
improving the skill of oceanographic models. Furthermore, dense grids
of sampled oceanographic data have the potential to fuel developments
targeting existing gaps in modeling skill when different levels of
spatial and temporal resolution are considered \cite{Balaji_2022}.


\begin{figure}[!]
  \centering
  \includegraphics[scale=0.3]{fig/iqd_3D.png}
  \caption{Trajectory of the XP5 during the 30 October mission in the
    \naz Canyon region. The color shading represents the interquartile
    range (IQD) of temperature values recorded within each model grid
    cell along the vehicle’s path. The grid illustrates the horizontal
    resolution of the statistical model, and the results are shown for
    three representative depth layers: near-surface, mid-depth (20 m),
    and 40 m, capturing distinct vertical regimes. Higher IQD values
    indicate regions of higher thermal variability within the model
    grid.}
  \label{fig:iqd_3D}
\end{figure}


