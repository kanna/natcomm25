\begin{itemize}
    \item 
    \item \section{Methods}
\end{itemize}
% max 3000 words

Our methodology is based on the implementation of a complete data cycle to enhance model predictions in a coastal ocean domain through adaptive sampling and data assimilation. The approach consists of three fundamental steps, executed iteratively on a daily basis:

\begin{enumerate}
    \item \textbf{Model Forecast and Uncertainty Projection}:  
    A numerical ocean model provides a daily one-step forecast $\hat{\theta}(k+1, x, y)$ of a target oceanic variable $\theta$, along with an associated uncertainty field $\sigma_{\hat{\theta}}(k+1, x, y)$, where $k$ represents the current day and $(x, y)$ denote the geographical coordinates. These outputs are organized into discrete spatial maps, $M_{\hat{\theta}}(x, y)$ and $M_{\sigma_{\hat{\theta}}}(x, y)$, representing, respectively, the predicted state and its uncertainty over a predefined grid covering the study area.
    
    \item \textbf{Target Sampling Planning}:  
    Using the uncertainty map $M_{\sigma_{\hat{\theta}}}(x, y)$ as input, an target sampling algorithm determines the set of trajectories for a fleet of $N$ Autonomous Underwater Vehicles (AUVs) for the next operational cycle. The goal is to maximize the accumulated uncertainty sampled along the vehicle paths, while satisfying vehicle-specific constraints such as maximum endurance, operational limits, and navigation feasibility. The planned trajectories are transmitted to the vehicles for execution.

    \item \textbf{Data Collection and Assimilation}:  
    Throughout the operational period, each AUV collects pointwise measurements of the target variable $\theta$ along its assigned path. After the mission is completed, the collected measurements are assimilated into the numerical model using an appropriate data assimilation scheme. This updated model state serves as the new initial condition for the next forecasting cycle, closing the loop.
\end{enumerate}

\subsection{Model Forecast and Uncertainty Projection}
Sampling strategies rely on timely information about the spatial variability of ocean properties. However, conventional numerical ocean models are computationally intensive, and their runtime makes them impractical for use in near-real-time mission planning. While similar assimilation cycles could in principle be implemented directly within deterministic numerical models, this would require either substantially higher computational resources or larger temporal horizons. As a more practical alternative, we employ geostatistical simulation as a computationally efficient surrogate, producing short-term predictions of ocean temperature together with estimates of uncertainty \cite{deutsch1992}. This approach captures the essential variability of local ocean dynamics at a fraction of the cost of full numerical models, while remaining flexible enough to assimilate new in situ observations quickly \cite{Duarte2025} and coherent with the scope of the work approach.

The methodology relies on ensembles of geostatistical realizations, each representing a state of the ocean temperature field conditioned by a priori deterministic ocean models \cite{CMEMS2017} and direct observations (FRESNEL fieldwork). By computing the pointwise standard deviation across the ensemble, we obtain spatial uncertainty maps that highlight regions where predictions are less constrained and potentially more informative for sampling. These maps serve as the basis for identifying areas where new measurements are expected to maximize the reduction of forecast error \cite{Duarte2025}.   

Geostatistical realizations are produced using direct sequential simulation \cite{soares2001direct}, a stochastic method that draws values from conditional probability distributions defined by kriging estimates and variances. The continuity of the temperature field in space and time is characterized by variogram models fitted to long-term calibrated ocean model data \cite{CMEMS2017}. At each depth, simulations are carried out independently, using a moving temporal window of fourteen previous days to predict the subsequent day. This sliding-window strategy strikes a balance between forecast skill and computational feasibility, and can be adapted according to the complexity of the oceanographic conditions.

The ensemble of realisations provides both a forecast of ocean temperature and a quantitative assessment of the prediction uncertainty. By updating the forecasts with new AUV measurements through sequential assimilation, the method progressively refines the temperature field while maintaining consistency with prior model dynamics. The resulting forecasts and uncertainty maps form the input for the target sampling algorithm, guiding the allocation of AUV trajectories towards regions of greatest expected information gain. More details about the model, its development, and application during the FRESNEL campaign can be found in \cite{Duarte2025}.

\subsection{Target Sampling Algorithm}

The adaptive sampling problem is formulated as the design of vehicle trajectories that maximize the amount of useful information extracted from a model-derived uncertainty map, while respecting the endurance and safety constraints of the fleet. Each day, the models provide both a forecast field and its associated uncertainty distribution, which together define the reward landscape for the planner. The task is then to generate, for each vehicle, a path composed of waypoints that accumulates the highest possible uncertainty values, subject to limits on distance, time, and operational feasibility.

To address this, the problem is cast in graph form. The spatial uncertainty map is first pre-processed to remove obstacles and smoothed to highlight large-scale features. Candidate waypoints are then identified from the map and used to build a weighted graph, where nodes carry a reward proportional to their uncertainty value and edges represent travel costs. The trajectory planning task is posed as a Prize Collecting Vehicle Routing Problem (PCVRP)\cite{vidal2013,toth2014vehicle}, a well-known formulation in combinatorial optimization where routes must balance the rewards obtained from visiting nodes with the costs of traveling between them.

By solving this problem, the algorithm returns a set of near-optimal trajectories that prioritize regions of greatest uncertainty, while ensuring vehicle endurance and safety constraints are respected. This provides a principled way of steering autonomous platforms toward the most informative sampling locations, forming a key component in the daily cycle of forecast, adaptive sampling, and data assimilation.

Additional insights into the algorithm formulation and its deployment during FRESNEL are reported in \cite{bernacchi2025}.

\subsection{Data Collection}

\subsubsection{System of systems for ocean observation}

TO DECIDE: 
\begin{itemize}
    \item INCLUDE IH explaining that here we are focused on auv data
    \item check for the definition of C2 for the first time
    \item PRESENT EXAMPLES OF RIPPLES LAYERS HERE OR REFER TO OTHER SECTIONS WERE THOSE MAY BE DISPLAYED
    \item INCLUDE MODEL AND MAKE OF SENSORS
\end{itemize}

The System of Systems for ocean observation deployed in this field study comprised systems contributed by the LSTS, the Hydrographic Institute (IH), Portuguese Navy, and the University of Columbia.

IH deployed the D. Carlos oceanographic vessel contributing wet laboratories, a rosette equipped with CTDs, Niskin bottles and an Underwater Vision Profiler %(http://www.hydroptic.com/index.php/public/Page/product_item/UVP6-LP)%,
and an Alseamar Glider equipped with a CTD %(https://www.alseamar-alcen.com/ocean-science-sector/seaexplorer-gliders/seaexplorer-1000/)%.

LSTS deployed 5 Xplore long endurance LAUVs  along with 6 Manta communication gateways (Figure \ref{fig:lauvs}) and control stations for the field experiments in Nazaré. The Xplore class LAUV \cite{lauvurl} 
is an UUV designed for water column operations equipped with CTD, fluorometer, turbidity, O2, cameras, and nutrient sensors, WiFi and satellite communications (Iridium), acoustic modems, and battery packs enabling 60h+ endurance. The CTDs mounted on the UUVs were calibrated and cross-calibrated at IH calibration facilities. The Manta communication gateways \cite{} 
are equipped with acoustic modems, satellite and Wi-Fi modules for long range communications with underwater, surface, and aerial vehicles. The UUV operations were supported by two boats rented in Nazaré. In addition, to the existing command and control center located in Porto, LSTS installed another in Nazaré. Two Manta units were installed in the support boats to support local deployments and communications with the command and control centers. Four units were used to support the command and control center installed in Nazaré.

\begin{figure}
    \centering
    \includegraphics[width=.7\linewidth]{fig/lauvs.png}
    \caption{The LAUVs deployed in the experiment.}
    \label{fig:lauvs}
\end{figure}


The University of Columbia deployed a vertical profiler equipped with a CTD and portable CTD deployed from the support boats.

The Xplore LAUVs, control stations, and Manta gateways were powered by the LSTS software toolchain \cite{pinto2013lsts} deploying command and control, data pipelines, and algorithm integration to support closing the modeling-sample-assimilation-tasking cycles. The 4 components of the toolchain are briefly described below:

DUNE, the onboard control software, running on vehicles and other devices. DUNE interfaces with sensors and actuators, by sending commands and monitoring them. DUNE is also in charge of executing missions and commands sent by the command and control (C2) stations. DUNE and its upper water column backseat implementation provided additional functionally for Iridium satellite communications, telemetry, and interaction with Neptus. 

IMC, the messaging protocol. It includes a definition of messages and its serialization. It is used to send and receive commands and data between vehicles and the C2 stations. A few IMC messages were created to support this field experiment.
NEPTUS, the C2 software framework. Neptus presents a graphical user interface (GUI) enabling the operator to connect to one or several heterogeneous vehicles (underwater, surface, aerial, or sensors). With Neptus, one operator can control multiple vehicles simultaneously, sending commands, missions, and receiving and monitoring telemetry and data using Wi-Fi acoustics, or satellite communications. For this deployment Neptus incorporated new developments to streamline the integration of telemetry coming through the Iridium satellite communications with existing GUIs and to import optimal routes generated by the adaptive sampling algorithm into mission plans to send to the vehicles. 

RIPPLES, the Web based C2 that supports situational awareness of unmanned systems operations. It supports real-time monitoring of an operation and planning of such operations by providing access to several data products layers. These layers include environmental data and other support models for the environment and additions marine traffic in the vicinity of the operating autonomous vehicles (such AIS and aircraft data, or density maps). New layers were added to RIPPLES to support new aspects planning and execution control for this deployment. 

Figure XXX shows outputs of the HOPS (Harvard Ocean Prediction System) with temperature and salinity model outputs and respective errors. Another layer displayed geostatistical model outputs XXX. These outputs were used to monitor progress by enabling the visualization of updated during the execution of the operation.

IH has several buoys in the area. A new layer to directly access and visualize their data in real-time was added to RIPPLES (Figure \ref{fig:buoys}. This provided wave information for the area (real-time and historical).


\begin{figure}
    \centering
    \includegraphics[width=.7\linewidth]{fig/buoys.png}
    \caption{The LAUVs deployed in the experiment.}
    \label{fibuoys}
\end{figure}

 
Optimal routes were generated for the UUVs to execute. A new Ripples layer enabled the user to visualize all the routes taken by the UUVs, as well as the routes generated for future deployments. This contributed to a timely launch and deployment of the vehicles.
 
New layers for the visualization of remote sensing data, as well as of data collected by sensors installed onboard NRP D. Carlos were also deployed.


% The adaptive sampling problem is formulated as an optimization task aiming to maximize the expected information gain from the observations. Specifically, given the predicted uncertainty map and operational constraints (e.g., time, energy budget), the algorithm plans vehicle routes that prioritize areas of higher model uncertainty. The uncertainty along each candidate path is evaluated, and path planning strategies — such as greedy heuristics or combinatorial optimization techniques — are employed to allocate waypoints efficiently among the available AUVs.

\subsubsection{Field deployment: planning and preparation}

IH contributed the NRP D. Carlos for 5 days during one of the planned offshore buoy maintenance missions taking place annually in April or in October. The October period was selected for the FRESNEL deployment. The meteorological and ocean conditions are typically not significantly different in these periods and may be affected by distant storms taking place in the Atlantic.

The planned time window for the FRESNEL deployment with NRP D. Carlos was the second week of October. However, this time window had to be shifted by one week, to start October 20th, because of challenging meteorological and ocean conditions. The UUV deployments were planned to start the first week of October but were delayed starting October 14th and end October 31st. The decisions to delay both the ship and UUV deployments proved to be adequate and made it possible to have 5 days of ship time and 7 days of UUV operations. This also enabled concurrent operations of the ship, glider, and UUVs. The UUVs collected data one week in advance of the ship’s arrival and kept running the developed algorithms during the third week of the deployment. The planned areas of operation for NRP D. Carlos (large rectangle) and for the UUVs (smaller rectangle) are depicted next. 

\begin{figure}
    \centering
    \includegraphics[width=.7\linewidth]{fig/Opareaas.png}
    \caption{Areas of operations for NRP D. Carlos and for the UUVs.}
    \label{fig:opareas}
\end{figure}

These areas have intense ship traffic, particularly from fishing vessels, that presents added collision risks to UUVs at surface or when surfacing, as well as potential encounters with bottom trawling fishing nets. In addition, because Nazaré is a fishing harbor there is the added challenge of UUV encounters with fishing nets.

To address these challenges we started by meeting fisherman associations with the support of the city hall to engage the community, understand their fishing procedures and the probable locations of fishing nets. Second, we studied AIS patterns and densities, first for one year, then for one month, and finally for each week in October. This enabled us to partition the operations area into 3 areas with increasing risk levels \ref{fig:riskareas}. We focused our operations on the first two areas, starting operations at the one where risk was lower. Finally, by studying daily patterns we were able to establish temporary areas for operations with acceptable risks. We have distilled this knowledge into operational procedures and automated decision aids and alerts.

\begin{figure}
    \centering
    \includegraphics[width=.7\linewidth]{fig/riskareas.png}
    \caption{The operational area was partitioned into 3 areas with different levels of risk} XXX PLACE RISK MAP SIDE BY SIDE
    \label{fig:riskareas}
\end{figure}

Field operations were organized into two concurrent activities:

\begin{itemize}
    \item NRP D. Carlos campaign taking place October 20th – 24th. Team members from the University of Aveiro were onboard NRP D. Carlos to support operations with the UVP and wet laboratory operations. In addition to the rosette casts a glider was also deployed from the ship. The deployment of the glider along the western boundary of the operations area provided some of the initial conditions for the HOPS model. The rosette casts provided a course description of essential ocean variables in the larger area of operation. XXX Detailed descriptions of these activities are presented in Annexes E - H.
    \item UUV deployments taking place October 14th - 31st.  These deployments were supported by two boats rented in Nazaré. In addition to the UUV deployments by LSTS-UPorto, water samples were collected by researchers from Columbia University. These data collection activities provided a dense grip of data collection points. XXX Detailed descriptions of these activities are presented in Annexes D, F, G.
\end{itemize}

These two activities were conducted in coordination, namely in what concerns water space management and sharing of model predictions, done with the help of the LSTS toolchain and the underlying communication infrastructure. Both activities run 24/x. In the case of the UUV deployments the LSTS-UPorto team operated in 4 6-hour long shifts with minimal operator’s footprint (one active and one backup operator). Operations were run from a house rented in Nazaré and from LSTS in Porto.


\subsubsection{Field deployment: summary of UUV and ship operations}

Questions:
\begin{itemize}
    \item explain yoyos
    \item include a more detailed description of daily ops with tables?
\end{itemize}


UUV daily operations started with the analysis of remote sensing data, data collected during the previous 24h, forecasts of meteorological and ocean condition, performance of the planning and execution control algorithms, performance of the modeling-sample-assimilation-tasking cycle (this was done during the last week of operations), and mission updates provided by the operators in charge of the night shifts.

UUV operations’ planning then proceeded in 2 different time horizons:
\begin{itemize}
\item Plan for the day, including launch and recovery of vehicles, as well as boat operations.
\item Plan for next 2-3 days (3 UUVs could operate for 50h+). This was particularly challenging because it included calculating time of recovery and making sure that the meteorological and ocean conditions were feasible for these operations.
UUV execution control was focused on addressing alerts (e.g., ships crossing the area of operations), coordinating launch and recovery of UUVs with the help of the two boats, and re-tasking the vehicles in case a significant change in the planning assumptions occurred. 
\end{itemize}

FRESNEL operations started spanned October 14th - 31st period. We took a risk-minimizing incremental approach to operations planning and execution control. The first week was about getting acquainted with the new area of operations and evaluating, testing, and improving operational procedures. The second week was about deploying the whole approach and learning from it. The third week focused on running UUV operations to further evaluate and test the approach, namely in what concerned simultaneous deployments and persistent observation.

The first week was about UUV and small boat operations (including collecting water samples). The focus was on testing the validity of the risk management approach (including the partition of the operations area) and on evaluating and testing the operation of the UUVs in this new area. UUV missions were already spanning at least 2-day durations.

The second week involved the 5-day long NRP D. Carlos campaign together with UUV operations with boat support for launch and recovery and water sampling. Coordination of these concurrent operations involved water space management, to prevent collisions, and ingestion of data provided by the ship and UUVs for analysis and assimilation. The experience acquired during this week was invaluable and provided the templates for UUV operations taking place the following week. Over these two weeks the UUVs were impacted several times by strong vertical currents in areas of significant stratification. Our preliminary analysis pointed to internal waves as the most probable cause of these currents. The area is known for internal wave activity and remote sensing data provided evidence of the presence of internal waves during the same days. 

The final week focused on UUV operations only. Multi-day UUV deployments provided additional data about the modeling-sample-assimilation-tasking cycle (executed several times). In addition, UUVs were deployed concurrently not only in the area surveyed the previous week (to minimize risk) but also in areas with higher risk of collisions in which short time windows limited the duration of these deployments. This required tighter planning and execution control procedures for the two operators in charge of the concurrent operations. This is illustrated with mission plans depicted in Figure \ref{fig:missionplans}. One UUV was tasked to perform yo-yos along straight line (line in black) in an area in which trawlers run North-South transects, while two others performed the adaptive sampling algorithm further South in a safer area.
 
\begin{figure}
    \centering
    \includegraphics[width=.7\linewidth]{fig/missionplans.png}
    \caption{Example of mission plans (tours) to be executed by 3 UUVs.}
    \label{fig:missionplans}
\end{figure}


The UUV operators were provided with close to real-time information about the data collected by the UUVs, as depicted in Figure \ref{fig:temperatureprofiles} showing surfacing points for one mission color-coded by temperature. The operator could click in any of these points to get a temperature profile for the previous dive.

 \begin{figure}
    \centering
    \includegraphics[width=.7\linewidth]{fig/temperatureprofiles.png}
    \caption{UUV path color-coded by levels of temperature at surfacing points.}
    \label{fig:temperatureprofiles}
\end{figure}



Despite the challenging meteorological and ocean conditions, that strongly constrained the overall deployment, the FRESNEL team was able to demonstrate the overall approach during these 3 weeks (out of which only 7 days of operation were possible). The team operating the UUVs spent these 3 weeks on site and on call to rapidly deploy or recover the UUVs as dictated by meteorological and ocean conditions or forecasts. 



\subsection{Field Experiment Analysis and Evaluation}

talk about how the analysis will be done (cases and methods to evaluate)


% The practical implementation of the data cycle in a real-world marine environment introduces several constraints:

% \begin{itemize}
%     \item \textbf{Communication Limitations}:  
%     Low-bandwidth and intermittent communications at sea require that mission planning be sufficiently robust to accommodate long periods of autonomous operation without human intervention.
    
%     \item \textbf{Vehicle Constraints}:  
%     AUV endurance, payload limitations, navigation precision, and deployment risks all impose restrictions on the feasible operational space and mission duration.

%     \item \textbf{Computational Demands}:  
%     Real-time generation of uncertainty fields, optimization of paths, and assimilation of collected data must be performed within time windows compatible with daily operational cycles, often under constrained computational resources.

%     \item \textbf{Environmental Variability}:  
%     Fast-evolving coastal ocean dynamics can introduce discrepancies between forecasted and actual conditions, necessitating robust planning that accounts for forecast uncertainty and adaptivity.
% \end{itemize}

% This method was deployed and evaluated under the operational framework of the FRESNEL project, providing a real-world demonstration of the data cycle’s feasibility and benefits in complex coastal environments.


% Distinction Between Onboard and Offboard Predictions:
% \begin{itemize}
%     \item Onboard: Statistical prediction
%     \item Offboard: Numerical prediction
% \end{itemize}

% Rationale and Implementation (including a schematic figure):
% \begin{itemize}
%     \item Why this approach?
%     \item How was it executed?
% \end{itemize}

% Adaptive Sampling Strategy:
% \begin{itemize}
%     \item contraints
%     \item Algorithm used for real-time decision-making
%     \item Criteria for data collection optimization
% \end{itemize}

% Statistical Approach:
% \begin{itemize}
%     \item Techniques applied for uncertainty estimation
%     \item Integration with observational data
% \end{itemize}

%  Numerical Model (HOPS - Harvard Ocean Prediction System):
% \begin{itemize}
%     \item Model configuration and setup
%     \item Data assimilation methods
% \end{itemize}


