\section{Conclusions}
\label{sec:conclude}

This effort demonstrates the overall integration to close the
\emph{sample-assimilate-predict-direct} cycle with the goal of
improving oceanographic model skill by leveraging observations from
robotic platforms. Figs. \ref{fig:rms_ABB1} and \ref{fig:rms} show
clear indications of increase in predictive skill when high resolution
data from AUVs are assimilated in a geostatistical model. By choosing
such a modeling approach, this effort offers a window into the rapid
process of assimilating and prediction, the primary goal of \proj with
and doing so with minimal robotic operation support. The accuracy of
the predictions of the geostatistical model increased after several
cycles, while the overall prediction errors decreased as noted.

Importantly, this effort puts us on a viable path towards continuous
ocean prediction when obtaining high resolution in-situ data as
conceptualized in the \met framework \cite{rajan21}.

While a number of challenges remain in our approach, including near
real-time assimilation and continuous model prediction especially to
capture dynamic coastal events, the dense grid of AUV observations
enables post-experiment evaluation and testing of the assimilation
schemes as well of the geostatistical model.  While we are still
lacking the desired statistical significance of a long series of
consecutive cycles, our future work will target the optimization of
the parameters used for deeper integration of the algorithms used in
the cycle. We will further investigate the selection of representative
depths for the application of the sampling algorithm used to find the
horizontal projection of the AUV paths.

% The \proj experiment carried out in challenging ocean and
% meteorological conditions precluded cycles of week-long operations, as
% originally planned. This effort nevertheless demonstrates the overall
% integration to close the \emph{sample-assimilate-predict-direct} cycle
% with the goal of improving oceanographic model skill by leveraging
% observations from robotic platforms. 


% Full integration of command and
% control strategies comprising modeling, assimilation and adaptive
% sampling was demonstrated with minimal operator support. The accuracy
% of the predictions of the geostatistical model increased after several
% cycles, while the overall prediction errors decreased as noted.

% However, these results still lack statistical significance
% because of the few consecutive cycles in which the overall approach
% was tested. We observed that we never had more than 3 consecutive days
% of operation.




