\section{Introduction}
\label{sec:intro}

The coastal ocean, the marine areas extending from the coastline to
the continental slope, encompassing the continental shelf, constitutes
one of the most dynamic and complex regions of the ocean. These areas
are characterized by high variability, intricate topographies, and
coastline geometries. They are forced by a wide range of physical and
biogeochemical processes which occur across diverse spatial and
temporal scales. They are among the most productive and economically
significant areas of the world’s oceans, benefiting from terrestrial
inputs via river discharges and nutrient renewal driven by upwelling
processes. As a result and in part, coastal areas concentrate a great
proportion of human activities including leisure, to fisheries and
offshore aquaculture to renewable energy exploitation. Moreover, the
coastal ocean acts as an interface between deep-ocean processes and
coastal environment, modulating how large-scale phenomena — such as
climate-driven changes or extreme weather events — impact coastal
populations. It also regulates, for example, how anthropogenic
influences originating on land are redistributed and affect marine
ecosystems \cite{greene25}.

The complex interplay between physical, chemical, and biological
processes in the coastal ocean can rapidly amplify or modulate the
impacts of extreme events such as storms, flooding, or pollution
incidents. Furthermore, the coastal ocean serves as a key interface
where large-scale oceanic and atmospheric changes manifest their
consequences for coastal ecosystems. Accurately forecasting the state
of the coastal ocean is therefore essential not only for safeguarding
environmental and economic interests, but also for enhancing
resilience to climate variability, natural hazards, and human-induced
pressures.

Ocean models have become essential tools for predicting ocean state
and underpin a wide range of societal applications, from climate
forecasting to pollution monitoring and resource management. However,
these models are inherently imperfect representations of complex
reality. Data assimilation techniques — where observational data are
integrated into models — play a critical role in correcting or
adjusting model errors and improving their forecasting
capabilities. Yet, the success of data assimilation depends heavily on
the availability, quality, and relevance of observations.

Satellite remote sensing offers extensive spatial coverage but is
limited to surface measurements with usually low spatial resolution
and often compromised by atmospheric conditions. In situ observations,
whether from ships, moorings, or drifting platforms, provide higher
accuracy but suffer from limited spatial and temporal coverage, high
operational costs, and logistical constraints — particularly high in
coastal regions. Recent technological advances have enabled the use of
autonomous underwater vehicles (AUVs) and other robotic platforms,
offering flexible, mobile, and increasingly autonomous means of
sampling ocean properties with high spatiotemporal resolution and low
logistic footprint
\cite{das10,das11b,olaya12,graham12,jdas13,das15,sousa16,fossum18,fossum19b}.


However, even with these new tools, efficiently gathering data at the
right spatial and temporal scales remains a complex challenge. AUVs
are constrained by endurance, communication bandwidth, and operational
risks such as vessel traffic, high speed currents and bathymetry. In
addition, the ocean itself remains highly dynamic with key processes
often evolving faster than traditional sampling strategies can
capture. This opens the door to adaptive sampling strategies, where
observation efforts are guided by model outputs and uncertainty
estimates to maximize the relevance of collected data. Adaptive
sampling \cite{BinZhang07,Singh09,smith14,fossum19} represents a
fundamental shift: instead of executing pre-planned missions,
autonomous vehicles dynamically prioritize areas for observation.

Our approach combines model predictions with adaptive AUV
sampling. Doing so combines predicted model uncertainties in
comparison with prior data assimilation process with available
observations. Model forecasts identify where uncertainty is highest or
where new observations are expected to have the greatest impact in
reducing forecast errors (i.e., higher value of information). Vehicles
are then directed to sample these areas, and their data are in turn
assimilated back into the models, as a virtuous cycle. Doing so closes
a critical feedback loop between data acquistion, models and
observations; model prediction generates a forecast and associated
uncertainty field while AUV based adaptive sampling targets areas of
maximum predicted uncertainty. Data assimilation from these high
uncertainty regions are then incorporated into the model such that new
predictions benefit from improved in-situ data, closing the loop.
Such a \emph{sample-assimilate-predict-direct} data cycle approach
promises to significantly enhance the skill of ocean forecasts,
especially in the complex coastal environment where processes occur
over a wide range of scales. Additionally, sampling resolution with
AUVs vastly outweighs data obtained from traditional ship-based
approaches. Multiple AUVs sampling simultaneously as we do here, also
makes a dent in spatial as well as temporal coverage of such dynamic
environments at far moderate costs. Further, AUVs can provide near
real-time data allowing for their rapid re-tasking and potentially
placement in spatially separated regions.

\begin{figure}[!t]
  \centering
  \includegraphics[scale=0.5]{fig/ensemble-2.jpg}
  \caption{\proj demonstrates the value of integrating ocean models
    with adaptive robotic vehicles in the coastal ocean, to increase
    model skill and prediction within a tight control loop.}
  \label{fig:block-diag}
\end{figure}

In this study, we address the challenge of such loop closure by
implementing and evaluating a complete data cycle — from model
prediction, to uncertainty projection, to adaptive sampling and data
assimilation — in a coastal ocean environment. The work was conducted
within the framework of \proj (\textbf{F}ield expe\textbf{R}iments for
mod\textbf{E}ling, a\textbf{S}similatio\textbf{N} and
adaptiv\textbf{E} samp\textbf{L}ing), specifically designed to explore
and test model-driven robotic exploration strategies
(Fig. \ref{fig:block-diag}). The experimental results obtained within
\proj serve to illustrate and validate the proposed approach,
highlighting the practical benefits and challenges of real-world
adaptive ocean exploration. \proj itself is part of a larger concept
for multi-domain sensing, observation and exploration, which couples
ocean models, autonomous robotics, Artificial Intelligence and Machine
Learning coupled with a small satellite constellation
\cite{rajan21}. Our end goal is to democratize ocean model prediction
by lowering costs, providing access to a range of stakeholders, by
integrating methods in low-cost computation of modeling services
coupled with spatio-temporal data assimilation from mobile or immobile
robots.

This manuscript is organized as follows. Section \ref{sec:intro}
provides an introduction to the problem we are tackling providing a
motivation for the \emph{sample-assimilate-predict-direct}
loop. Section \ref{sec:study-area} provides the environmental context
and the challenge of modeling and operating robotic vehicles in a
dynamic coastal region such as \naze. Section \ref{sec:methods}
highlights the details of the forecasting, assimilation, sampling and
operational strategies which lie at the core of this paper, while
Section \ref{sec:disc} discusses the results from the experiment with
conclusions and future work in Section \ref{sec:conclude}.
