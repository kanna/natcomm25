\section{Conclusions}
\label{sec:conclude}


This effort demonstrates the overall integration to close the
\emph{sample-assimilate-predict-direct} cycle to improve oceanographic
model skill by leveraging observations from robotic
platforms. Figs. \ref{fig:rms_ABB1} and \ref{fig:rms} show clear
indications of an increase in predictive skill when high-resolution
data from AUVs are assimilated in a geostatistical model. By choosing
such a modeling approach, this effort offers a window into the rapid
process of assimilating and predicting, the primary goal of \proje,
and doing so with minimal operational support. The accuracy of the
predictions of the geostatistical model increased after several
cycles, while the overall prediction errors decreased as noted. These
results provide a foundation for continuous ocean prediction when
obtaining high-resolution in-situ data as conceptualized in the \met
framework \cite{rajan21}.

A confluence of external factors reduced the observation period from
the planned three weeks to three days, primarily due to weather,
platform availability, and personnel constraints.  Future work will
extend the observation period to validate and expand upon these
promising results.

Other challenges that remain to be addressed include near real-time
assimilation and continuous model prediction especially to capture
dynamic coastal events.  While we are still lacking the desired
statistical significance of a long series of consecutive cycles, our
future work will target the optimization of the parameters used for
deeper integration of the algorithms used in the cycle. For instance,
we hope to investigate the selection of representative depths for the
application of the sampling algorithm used to find the horizontal
projection of the AUV paths \cite{bernacchi2025}. Another potential
outcome to be investigated is the use of higher-resolution numerical
models and longer prediction horizons, considering computational
trade-offs. In doing so, we would also like to demonstrate the
viability of \emph{portable} low (computational) cost models running
in the cloud, which can be initialized for any region rapidly to
demonstrate the loop-closure we set out to validate.

Along these lines, an additional research direction involves
encapsulating a model surrogate embedded within the control system of
one or more AUVs to capture coastal dynamism at fine scales
\cite{frolov09,fossum19b}, while complementing an increase in the
assessment of shore-based model skill.




