\section{Discussion}
\label{sec:disc}

Understanding the impact of targeted AUV observations on short-term
ocean forecasts requires examining not only overall model–data
agreement but also how assimilation effects vary across vehicles and
distinct dynamical environments. While the general trend indicates
that assimilation improves model skill, vehicle-specific results
reveal important differences tied to the spatial distribution of
sampling relative to coherent regions.

Figs. \ref{fig:rms_c} and \ref{fig:rms_d} present the RMSE predictions
compared with in-situ data under the different assimilation scenarios
(\textbf{C}, \textbf{C4}, \textbf{D}, and \textbf{D4}) for
31\textsuperscript{st} October. The results are shown as a function of
depth and vehicle, allowing a direct comparison with and without data
assimilation.  For XP3 and XP5, a clear and consistent trend is
observed: the assimilation scenarios (\textbf{C4} and \textbf{D4})
improves the model prediction skill, with RMSE decreasing across all
depths, indicating that the assimilation procedure effectively
constrains the local dynamics represented in the geostatistical
model. Although small differences are present between \textbf{C4} and
\textbf{D4}, they are not statistically significant and confirm that
assimilation results are robust with respect to initial configuration
differences.

The situation for XP2 was different however. The assimilation
scenarios (\textbf{C4} and \textbf{D4}) lead to higher prediction
errors compared with the non-assimilation runs (\textbf{C} and
\textbf{D}). This counterintuitive result stems from an important
spatial mismatch: on 31\textsuperscript{st} October, XP2 operated
north of the \naz Canyon, whereas the assimilated data from the
previous days (29\textsuperscript{th} and 30\textsuperscript{th}
October) were collected south of it. The \naz Canyon marks a
transition zone between distinct oceanographic regimes, where
circulation patterns are strongly influenced by the canyon’s complex
topography and the interaction between shelf and slope
processes. Inside the canyon, residual currents are generally aligned
along the canyon axis as a result of strong topographical control and
this alignment extends well above the canyon edges ($\sim150$m depth),
implying a substantial disturbance of the predominant north–south
circulation parallel to the general trend of the shelf and slope
\cite{tyler2009europe,relvas2007physical,guerreiro2014influence}. 

A broader interpretation of this result is provided by the CMS fields
shown in Fig. \ref{fig:north_south}, which compare temperature
distributions in the northern and southern regions of the study area
during the days preceding and during the experiment. Before
26\textsuperscript{th} October, although mean temperatures differ
between the two regions, their variability is consistent, indicating
that the system was evolving coherently across the canyon. After the
26\textsuperscript{th}, however, the temperature evolution in the two
regions diverge substantially, with the CMS model showing increasingly
de-correlated. This loss of coherence coincides with the onset of the
upwelling–relaxation transition and highlights a shift toward slightly
distinct dynamical regimes on either side of the canyon. This
transition can explain why assimilation of southern data—collected
under a single dynamical regime—fails to improve and, in fact,
degrades forecasts.

This spatial decoupling between sampling and assimilated information
also highlights the potential value of trajectory-planning strategies
that explicitly seek to maximise class separation among distinct
dynamical regimes, as proposed by \cite{fossum19b} using hierarchical
clustering methods. Such approaches, based on criteria such as
expected reduction of variance, mutual information, or entropy, could
help identify routes that better discriminate between competing
oceanographic states and mitigate situations where assimilation may
propagate misleading information if relying solely on uncertainty
fields.  

\begin{figure}[!]
  \centering
  \includegraphics[scale=0.2]{fig/north_south.png}
  \caption{Temperature distributions from CMS for the northern and
    southern regions of the study area between the
    18\textsuperscript{th} and 31\textsuperscript{th} October. The
    division between regions follows the line boundary shown in the
    inset map, top right. Only grid points shallower than 200m were
    retained to avoid including the canyon interface. Boxplots show
    the spatial distribution of CMS temperature within each region,
    capturing both the median and the spread of variability.}
  \label{fig:north_south}
\end{figure}

Another highlight of this experiment was in the visibility of
variability within the model grid providing additional insight into
the subgrid-scale structure of the observed
field. Fig. \ref{fig:iqd_3D} depicts the interquartile distance (IQD)
of measurements recorded by the AUVs within the three-dimensional
discretization of the statistical model. Each grid cell represents the
maximum spatial resolution available in the model, while the IQD
quantifies the local variability of in-situ measurements acquired as
the vehicles navigated through each cell.

This analysis demonstrates that robotic platform can capture
significant variability at scales smaller than the model resolution,
revealing the existence of fine-scale gradients, in some locations
greater than 1$^{\circ}$C, as also processes that are unresolved by
the model. Although the statistical model provides smoothed,
grid-averaged predictions of ocean properties, AUV measurements
highlight fluctuations that occur within individual cells, indicating
the presence of subgrid-scale dynamics. Such unresolved variability is
a well-known source of error in data assimilation systems; numerical
forecasts and statistical surrogates inevitably smoothen or filter
such variability occurring below their grid scale, so observations
that resolve submesoscale or turbulent features will often disagree
locally with grid-averaged model fields
\cite{oke2008representation,janjic2018representation}. These
considerations underscore that high-resolution observations from AUVs
should not be interpreted solely through coarse model fields. Instead,
subgrid variability itself provides valuable information about local
forecast uncertainty and should be explicitly accounted for when
assimilating observations into coarser-resolution predictive systems.

Across all missions, the AUVs sampled each grid cell with median
values ranging from 9 to 14 measurements per cell and mean values
between 19 and 26. These statistics indicate a relatively uniform and
dense sampling pattern, sufficient to characterize the local
variability of the water column at the subgrid level. The maximum
number of samples per grid cell reached several thousand, though these
values are biased by vehicle behavior near the surface, to
communicate and geolocate.

Despite these localized biases, the overall sampling density achieved
by the AUVs provide a detailed depiction of fine-scale ocean
variability. The IQD distributions reveal higher variability near the
thermocline, where vertical gradients are strongest, while in some
instances increased variability could also appear near the surface,
potentially corresponding to small-scale frontal zones or localized
mixing events. Because AUVs sample continuously along their
trajectories, they resolve spatial and temporal scales that often fall
between CTD stations in traditional ship-based surveys, thereby
capturing subgrid processes that remain unresolved in numerical
models.

Importantly, this subgrid variability is not merely observational
noise but a meaningful indicator of local uncertainty and
environmental heterogeneity. Incorporating both the variability
(i.e. IQD) and the spatial sampling density into data assimilation
frameworks enhances the model’s ability to represent uncertainty at
the subgrid scale and improve the realism of short-term forecasts.

This field study provided invaluable lessons in operational
procedures, refinement of adaptive sampling and algorithms, and risk
minimization to operate in an area with dense ship traffic and fishing
nets. Our AUVs were occasionally impacted by strong vertical currents
that may have resulted from the impact of internal waves that are
common in the area, and that were observed with the help of remote
sensing imagery during the experiment. Finally, the results achieved
with this deployment provided additional insights and the motivation
to further advance the state of the art in refining the
\emph{sample-assimilate-predict-direct} cycle with the goal of
improving the skill of oceanographic models. Furthermore, dense grids
of sampled oceanographic data have the potential to fuel developments
targeting existing gaps in modeling skill when different levels of
spatial and temporal resolution are considered \cite{Balaji_2022}.


\begin{figure}[!]
  \centering
  \includegraphics[scale=0.3]{fig/iqd_3D.png}
  \caption{Trajectory of the XP5 during the 30 October mission in the
    \naz Canyon region. The color shading represents the interquartile
    range (IQD) of temperature values recorded within each model grid
    cell along the vehicle’s path. The grid illustrates the horizontal
    resolution of the statistical model, and the results are shown for
    three representative depth layers: near-surface, mid-depth (20 m),
    and 40 m, capturing distinct vertical regimes. Higher IQD values
    indicate regions of higher thermal variability within the model
    grid.}
  \label{fig:iqd_3D}
\end{figure}


