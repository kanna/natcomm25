\subsection{Introduction}

Coastal ocean areas - the marine areas that extend from the coastline to the continental slope, comprising the totality of the continental shelf - constitute extremely dynamic regions that are forced by a large variety of agents. They are characterized by complex topography and coastline boundary and host a broad set of processes operating at
diverse spatial and temporal scales. They are among the most
productive and commercially important regions of the ocean, benefiting from the combination of terrestrial inputs of organic and inorganic material from river discharges and the renewal of nutrients from the upwelling of deep water. Not surprisingly coastal ocean areas accommodate the vast majority of human activities related to the sea, including major economic activities such as fisheries, offshore
aquaculture and wind energy. The coastal ocean shapes the two-way interaction between the deep ocean/ocean basins and the coastal populations and human societies. They determine, for example, how extreme events or responses to low period and global changes are transmitted from the deep ocean to coastal populations, and how anthropogenic influences originating from the continents are
redistributed, while impacting the maritime environment. % This explains
% why these regions are more directly affected by anthropogenic
% pressures.
 
% To understand and ultimately predict the evolution of the different
% processes that take place in the coastal ocean, is of vital importance
% to protect human life at sea and on the coast, to support the blue
% economy and to manage and preserve this rich marine environment.

% This environment also plays a significant role in security and defence
% needs of all nations. This importance becomes particularly evident to
% policy makers and public opinion during significant events such as
% those that affected the Portuguese coast during the last two
% decades. In the Entre-os-Rios tragedy (March 2001)
% \footnote{\url{https://en.wikipedia.org/wiki/Hintze_Ribeiro_disaster}},
% caused by the collapse of the Hintze Ribeiro bridge across the Douro
% River located more than 30 km inland from the coastline, the
% combination of strong river discharge and strong shelf circulation
% promoted by downwelling winds led to an astonishing rapid transport of
% victims' bodies along the coast, covering more than a thousand
% kilometers to reach the northern Spanish and the French coastlines.

During the \emph{Prestige} oil
spill\footnote{\url{https://en.wikipedia.org/wiki/Prestige_oil_spill}},
in November-December 2002, the slope % intensified flow that develops
% along the continental slopes of Western Portugal-Northern Spain
distributed the oil spill that resulted from the breaking and sinking
of the tanker which extended for over one thousand kilometers of the
coastlines of Spain, Portugal and France, transforming a local problem
into a European-wide crisis. More recently the conditions over the
coastal ocean domain determined the final evolution of Hurricane
Leslie (13-14 October 2018) and the impacts on the Portuguese
mainland. Such events highlight the importance of assimilation models
that are fed with actual observations to provide forecasts consistent
with the evolution of coastal ocean conditions and the crucial role of
subsurface conditions and processes.
% \item the rapid adjustment of coastal ocean conditions to changes in
%   forcing agents leading to significant changes of conditions after
%   one or two days
% \end{itemize}

In these events, as well as in many routine activities such as support
for Search And Rescue operations, % to Blue Economy sectors or Rapid
environmental assessment for naval operations, the capacity to observe
the actual conditions that affect the water column over the coastal
ocean regions of interest in such a way that these observations can
effectively feed operational models with assimilation, play a critical
role. This remains a challenge not only due to the large diversity of
processes operating at many different spatial and temporal scales in
the coastal ocean but also due to the concentration of human
activities which can dramatically constraint or render unviable,
monitoring activities.

% Not surprisingly these constraints translate to serious limitations in
% the use of assimilation in operational forecasting of coastal ocean
% areas. In the European landscape, for example, a recent survey
% promoted among members of \texttt{EuroGOOS} and its related network of
% Regional Operational Oceanographic Systems \cite{capet2020} showed the
% limited use of data assimilation in European forecast
% centers. Assimilation is largely restricted to physical variables and,
% among these, to surface observations collected by satellites and water
% column data collected by \texttt{ARGOS} profiles and, to a lesser
% extent to observations collected during regular programs (fixed
% platforms, cruises, glider lines). Assimilation of surface currents,
% for example collected by HF radar systems, is still very limited. The
% same study also highlighted the limited use of biogeochemical models
% (BGC) and variables.
 
% Existing operational forecast systems are frequently based on a number
% of assumptions that greatly limit the capacity of these models to
% describe the actual conditions observed in a given coastal ocean
% domain of interest at a given window of time. For example, many of
% these models incorporate the influence of freshwater discharges in the
% coastal ocean environment in the form of climatological monthly mean
% values associated with major rivers \cite{marta012}.  In operational
% support scenarios, these models can substantially diverge from
% real-world conditions which can be dictated by discharges from major
% rivers that substantially differ from the climatological picture or by
% large influence of small rivers and fresh water sources which are not
% included in the forecast models.
 
% These challenges are even more daunting in BGC models, especially
% those used in operationally. Components of these models are typically
% defined in terms of variable dependencies and parameters that are
% derived from climatology or based on a limited set of past
% observations. Assimilation in BGC models is largely restricted to the
% use of surface distributions of chlorophyll provided by satellite
% observations combined by established dependencies to infer other
% variable distributions. As a consequence the picture described by
% these models is frequently far from conditions actually observed and
% the capacity to respond to operational requests remains limited (see
% for example Figs. 5 and 6 from \cite{marta012}).
 
% Improving the ability of a BGC model to characterize and forecast the
% phyto- and zoo-plankton distributions in the water column is not only
% key to understanding the main processes shaping the coastal ocean
% ecosystems but essential to consistently forecast the development of
% algal blooms impacting coastal fisheries or to improve the capacities
% of underwater visibility models to support diver operations and beach
% access to the public.
 
% \begin{quote}
%   \textsf{General Objective 1:} \proj will explore and evaluate
%   strategies to build the initialization and assimilation fields to be
%   used by BGC models applied to a coastal ocean area. The approach
%   would be similar to the overall strategy used in the context of
%   Rapid Environmental Assessment for Navy operations and will leverage
%   observations from autonomous vehicles, traditional methods such as
%   ship-based measurements and opportunistic measurements using low
%   cost sensors, exploring synergies between dedicated surveys and
%   opportunistic observations.
% \end{quote}

\proj (\textbf{F}ield expe\textbf{R}iments for mod\textbf{E}ling,
a\textbf{S}similatio\textbf{N} and adaptiv\textbf{E}
samp\textbf{L}ing) was designed to explore and evaluate strategies to
for model driven exploration, by building assimilation into ocean
models to project entropy measurements as a mean to target adaptive
sampling which in turn would improve ocean model predictive skills.
By improving the capacity to observe and forecast a coastal ocean
domain of interest the approach was meant to significantly contribute
to improving our understanding of how the combination of physical and
biogeochemical processes may impact the coastal ocean. And in doing so
it articulates a broad range of observation systems in combination
with data assimilation models to acquire insight into the integrated
physical and biogeochemical processes that are associated with the
coastal ocean environment.

% Critical impact
% can occur in those areas where the prevailing physical processes can
% promote a rapid intake of nutrients into the upper layers that boost
% rapid development at all levels of the trophic chain. In the coastal
% ocean environment those important vertical motions linking the surface
% conditions with the subsurface processes can be promoted by the
% presence of the coastal boundary, by the complex topography of the
% continental shelf and slope or by the vertical mixing that occurs at
% different scales (e.g cold water cascade events at regional scale,
% vertical mixing associated with solitons at submesoscale). They are
% particularly expressive in the areas of the coastal ocean cut by
% narrow submarine canyons. These deep incisions of the continental
% slope extending over the shelf are a ubiquitous features of coastal
% ocean areas.  They interact with the wind driven shelf circulation
% promoting for example, a rapid and intensified response to upwelling
% favorable winds \cite{she00} which, in the case of a long submarine
% canyon, can be sustained in time despite the variability, leading to a
% persistent upwelling condition \cite{allen00}.

% \end{quote} 