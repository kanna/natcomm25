\section{Study Area}

The study was conducted in the coastal ocean off central Portugal,
focusing on the region influenced by the \naz Canyon (39.2$^{\circ}$
-39.9$^{\circ}$N). This area is shaped by strong topographic contrasts,
including the transition from the wide Estremadura Plateau to the
narrower shelf to the north, the long and narrow \naz submarine canyon
that incises the shelf and extends more than 200 km offshore, and the
Berlengas archipelago, a UNESCO Biosphere Reserve with high ecological
value. These features contribute to enhanced biological productivity and
biodiversity, and strongly modulate physical and biogeochemical
processes in the region.

Freshwater inputs from major rivers, such as the Tagus, have only a
limited direct impact on the area. In contrast, smaller rivers and the
\'{O}bidos lagoon can episodically deliver low-salinity, nutrient-rich
plumes to the shelf. Circulation is controlled by seasonal wind forcing
associated with the Azores High, with persistent upwelling-favorable
northerly winds in summer and frequent downwelling episodes in winter
under southerly winds. The interplay between canyon topography, shelf
circulation, and atmospheric forcing generates complex mesoscale
dynamics, intensified tidal currents, and internal wave activity that
promote strong vertical mixing and cross-shelf exchanges
\cite{martins10,quaresma07}.

The combination of sharp bathymetric gradients, variable forcing, and
rich physical–biogeochemical interactions makes the \naz Canyon region
an ideal natural laboratory to test adaptive observation strategies. In
particular, its dynamic environment poses both opportunities and
challenges for the \proj experiment, providing a representative coastal
setting in which to evaluate how model-based uncertainty projections can
guide adaptive sampling and assimilation to improve ocean model
predictive skill.
