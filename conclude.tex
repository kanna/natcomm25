\section{Conclusions}
\label{sec:conclude}

This effort demonstrates the overall integration to close the
\emph{sample-assimilate-predict-direct} cycle with the goal of
improving oceanographic model skill by leveraging observations from
robotic platforms. Figs. \ref{fig:rms_ABB1} and \ref{fig:rms} show
clear indications of increase in predictive skill when high resolution
data from AUVs are assimilated in a geostatistical model. By choosing
such a modeling approach, this effort offers a window into the rapid
process of assimilating and prediction, the primary goal of \proje,
and doing so with minimal operational support. The accuracy of the
predictions of the geostatistical model increased after several
cycles, while the overall prediction errors decreased as noted.
Importantly, this effort puts us on a viable path towards continuous
ocean prediction when obtaining high resolution in-situ data as
conceptualized in the \met framework \cite{rajan21}.

A confluence of factors well outside our control, influenced our short
three day observation cycle down from a planned three week experiment,
primarily driven by weather, availability of platforms and
personnel. Future work will extend this period of observation to
gather more data from the promising results presented.

Other challenges that remain to be addressed include near real-time
assimilation and continuous model prediction especially to capture
dynamic coastal events. % the dense grid of AUV observations enables
% post-experiment evaluation and testing of the assimilation schemes as
% well of the model.
While we are still lacking the desired statistical significance of a
long series of consecutive cycles, our future work will target the
optimization of the parameters used for deeper integration of the
algorithms used in the cycle. For instance, we will investigate the
selection of representative depths for the application of the sampling
algorithm \cite{bernacchi2025} used to find the horizontal projection
of the AUV paths. Another potential outcome to be investigated is the
use of a higher resolution numerical model and the implication of a
longer lead time to prediction given the challenges of using larger
computational resources.  In doing so, we would also like to
demonstrate the viability of \emph{portable} low (computational) cost
models running in the cloud, which can be initialized for any region
rapidly to demonstrate the loop-closure we set out to validate.

Along these lines, one interesting direction is in encapsulating a
model surrogate embedded within the control system of one or more AUVs
to capture coastal dynamism at fine scales \cite{frolov09,fossum19b},
while complementing an increase in shore based model skill assessment.


% Full integration of command and
% control strategies comprising modeling, assimilation and adaptive
% sampling was demonstrated with minimal operator support. The accuracy
% of the predictions of the geostatistical model increased after several
% cycles, while the overall prediction errors decreased as noted.

% However, these results still lack statistical significance
% because of the few consecutive cycles in which the overall approach
% was tested. We observed that we never had more than 3 consecutive days
% of operation.




