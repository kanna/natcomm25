\section{Conclusions}
\label{sec:conclude}

The \proj experiment carried out in challenging ocean and
meteorological conditions precluded cycles of week-long operations, as
originally planned. Nevertheless, this effort demonstrates the overall
integration to close the \emph{sample-assimilate-predict-direct} cycle
with the goal of improving oceanographic model skill by leveraging
observations from robotic
platforms. % traditional methods such as ship-based measurements and
% opportunistic measurements using low-cost sensors, and by exploring
% synergies between dedicated surveys and opportunistic observations.

Full integration of command and control strategies comprising
modeling, assimilation and adaptive sampling was demonstrated with the
help of the mature software toolchain enabling 24X7 operations with
minimal operator’s support. The deployments were performed to evaluate
and test the overall approach in an incremental fashion. The last days
of the deployment demonstrated several iterations of the closed
modeling-sampling-assimilation-tasking cycle. The accuracy of the
predictions of the geostatistical model increased after several
cycles, while the overall prediction errors decreased. However, these
results still lack statistical significance because of the few
consecutive cycles in which the overall approach was tested. We
observed that we never had more than 3 consecutive days of operation.
