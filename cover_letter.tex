\documentclass[10pt,stdletter]{newlfm}
\usepackage{charter}
\usepackage{times}

\widowpenalty=1000
\clubpenalty=1000

\newsavebox{\Luiuc}
\sbox{\Luiuc}{%
	\parbox[b]{1.75in}{%
		\vspace{0.5in}%
		\includegraphics[scale=0.8]
		{fig/uporto.png}%
	}%
}%
\makeletterhead{Uiuc}{\Lheader{\usebox{\Luiuc}}}

\newlfmP{headermarginskip=10pt}
\newlfmP{sigsize=10pt}
\newlfmP{dateskipafter=30pt}
\newlfmP{addrfromphone}
\newlfmP{addrfromemail}
\PhrPhone{Phone}
\PhrEmail{Email}

\lthUiuc

\namefrom{\vspace{-1.2cm}Renato Mendes}
\addrfrom{LSTS, University of Porto\\
	Rua Dr. Roberto Frias s/n, sala I203/4\\
	Porto, Portugal
}
\phonefrom{(+47) 98 678 309}
\emailfrom{\emph{rpmendes@inegi.up.pt}}

\addrto{%
Editor-in-Chief\\
Oceanography\\
}

\greetto{Dear Editor,}
\closeline{Yours faithfully,}

\Lfooter{University of Porto, Portugal\\
Underwater Systems and Technology Laboratory\\
4200-465 Porto, Portugal \\
}

\Rheader{1 of 1}

\begin{document}
\begin{newlfm}

  Please see our manuscript % \textbf{Towards Autonomous Adaptive Sampling
  %   of Phytoplankton in the Coastal Ocean} submitted for possible
  % publication in \emph{Science Robotics}. The overall context of the
  % work is in oceanographic Sampling, using an autonomous robotic
  % platform for upper water-column observation. The work is novel in
  % multiple ways. First, it uses Gaussian Process theory to track and
  % map phytoplankton biomass concentrations (measured as Chlorophyll
  % \emph{a}) in a 3D volume of water with the autonomous robotic
  % platform, in characterizing an important scientific problem. Second,
  % the work is verified and supplemented with multiple sources of data,
  % from buoys, remote sensing, and ship based observations (including
  % particle imaging systems) which contributes to demonstrating a novel
  % perspective for conducting interdisciplinary coastal ecological
  % studies, while combining new techologies to achieve a detailed
  % environmental picture of the water-column.

  % The article is a result of field work conducted in the coastal ocean and
  % includes in its authorship, biological and physical oceanographers,
  % as well as technologists and theorists.

  % The method is shown to successfully map and track the layer of high
  % Chlorophyll \emph{a} concentration, rendering a high resolution of
  % distribution within the water-column. The results point to effective
  % and practical benefits from leveraging methods from Statistics (GPs)
  % with robotic autonomy to simplify and focus sampling resources in a
  % volume of water.

  We believe such a work to be representative of new methods and
  approaches which are necessary and critical to understand our
  changing oceans.

\end{newlfm}
\vspace{-1cm}
\end{document}
