\section{Introduction}
\label{sec:intro}

The coastal ocean, the marine areaS extending from the coastline to
the continental slope, encompassing the continental shelf, constitutes
one of the most dynamic and complex regions of the ocean. These areas
are characterized by highly variable forcing mechanisms, intricate
topographies, and coastline geometries. They are forced by a wide
range of physical and biogeochemical processes which occur across
diverse spatial and temporal scales. They are among the most
productive and economically significant areas of the world’s oceans,
benefiting from terrestrial inputs via river discharges and nutrient
renewal driven by upwelling processes. As a result and in part,
coastal areas concentrate a great proportion of human maritime
activities, from fisheries and offshore aquaculture to renewable
energy exploitation. Moreover, the coastal ocean acts as an interface
between deep-ocean processes and coastal environment, modulating how
large-scale phenomena — such as climate-driven changes or extreme
weather events — impact coastal populations. It also regulates, for
example, how anthropogenic influences originating on land are
redistributed and affect marine ecosystems \cite{greene25}.

The complex interplay between physical, chemical, and biological
processes in the coastal ocean can rapidly amplify or modulate the
impacts of extreme events such as storms, flooding, or pollution
incidents. Furthermore, the coastal ocean serves as a key interface
where large-scale oceanic and atmospheric changes manifest their
consequences for coastal ecosystems. Accurately forecasting the state
of the coastal ocean is therefore essential not only for safeguarding
environmental and economic interests, but also for enhancing
resilience to climate variability, natural hazards, and human-induced
pressures.  Achieving this, however, remains a major scientific and
technological challenge, given the high variability, rapid dynamics,
and observational constraints inherent to these regions.

Ocean models have become essential tools for predicting the ocean
state and underpin a wide range of societal applications, from climate
forecasting to pollution monitoring and resource management. However,
these models are inherently imperfect representations of complex
reality. Data assimilation techniques — where observational data are
integrated into models — play a critical role in correcting or
adjusting model errors and improving their forecasting
capabilities. Yet, the success of data assimilation depends heavily on
the availability, quality, and relevance of observations.

Observations of the ocean are obtained through multiple sources.
Satellite remote sensing offers extensive spatial coverage but is
limited to surface measurements with usually low spatial resolution
and often compromised by atmospheric conditions. In situ observations,
whether from ships, moorings, or drifting platforms, provide higher
accuracy but suffer from limited spatial and temporal coverage, high
operational costs, and logistical constraints — particularly high in
coastal regions. Recent technological advances have enabled the use of
autonomous underwater vehicles (AUVs) and other robotic platforms,
offering flexible, mobile, and increasingly autonomous means of
sampling ocean properties with high spatiotemporal resolution and low
logistic footprint
\cite{das10,das11b,olaya12,graham12,jdas13,das15,sousa16,fossum18,fossum19b}.

However, even with these new tools, efficiently gathering data at the
right spatial and temporal scales remains a complex challenge. AUVs
are constrained by endurance, communication bandwidth, and operational
risks such as vessel traffic, high currents and bathymetry. In
addition, the ocean itself remains highly dynamic with key processes
often evolving faster than traditional sampling strategies can
capture. This opens the door to adaptive sampling strategies, where
observation efforts are guided by model outputs and uncertainty
estimates to maximize the relevance of collected data. Adaptive
sampling \cite{BinZhang07,Singh09,smith14,fossum19} represents a
fundamental shift: instead of executing pre-planned missions,
autonomous vehicles dynamically prioritize areas for observation.

One approach that we articulate here is combining model predictions
with adaptive AUV sampling. Doing so combines predicted model
uncertainties in comparison with prior data assimilation process with
availabe obervations. Model forecasts identify where uncertainty is
highest or where new observations are expected to have the greatest
impact in reducing forecast errors. Vehicles are then directed to
sample these areas, and their data are in turn assimilated back into
the models, generating improved forecasts for subsequent adaptive
planning as a virtuous cycle. Doing so closes a critical feedback loop
between models and observations; model prediction generates a forecast
and associated uncertainty field while adaptive sampling targets areas
of maximum predicted uncertainty. Data assimilation from these high
uncertainty regions are then incorporated into the model such that new
predictions benefit from improved in-situ data, closing the loop.
Such a \textit{data cycle} approach promises to significantly enhance
the skill of ocean forecasts, especially in the complex coastal
environment where processes occur over a wide range of scales and
where human and environmental stakes are high. It also has the
potential to improve operational efficiency by optimizing the
deployment of costly and resource-constrained observational assets. 
WE MAY WANT TO EXPLAIN THAT THIS CAN ONLY BE DONE WITH AUVS AND AT A MODERATE COST IN COMPARISON TO TRADITIONAL SAMPLING METHODS


IT MAY ALSO BE A GOOD IDEA TO HAVE A SHORT PARAGRAPH OM INNOVATION CLAIMS: something like

This has been done before, but we innovate in several aspects: 
\begin{itemize}
    \item Near-real time data visualization enable real.time response by operators..
    \item Simultaneous deployments in geographically separate subareas in the area of operations.
    \item Coordination also included risk minimizing considerations. This enabled operations in tight time intervals to collect data where it was needed taking advantage of openings provided by trawlers between consective legs.
    \item Parameterized framework enabling fine tuning of the overall approach. 
    \item Dense grid of observations.
    \item The sampling algorithm is not based only on position, but on error distributions and closed tours to maximize area coverage and utilization of the auvs.
    \end{itemize}

WHAT DO YOU THINK?

\begin{figure}[!]
  \centering
  \includegraphics[scale=0.5]{fig/ensemble-2.jpg}
  \caption{\proj demonstrates the value of integrating ocean models
    with adaptive robotic vehicles in the coastal ocean, to increase
    model skill while increasing model accuracy and prediction within
    a tight control loop.}
  \label{fig:block-diag}
\end{figure}



In this study, we address the challenge of such loop closure by
implementing and evaluating a complete data cycle — from model
prediction, to uncertainty projection, to adaptive sampling and data
assimilation — in a coastal ocean environment. The work was conducted
within the framework of \proj (\textbf{F}ield expe\textbf{R}iments for
mod\textbf{E}ling, a\textbf{S}similatio\textbf{N} and
adaptiv\textbf{E} samp\textbf{L}ing), a project specifically designed
to explore and test model-driven robotic exploration strategies. \proj
provided the experimental setting and operational assets to
demonstrate how model-based uncertainty fields can drive adaptive
sampling missions aimed at improving ocean model predictive skills to
enhance ocean forecasts in highly dynamic coastal environments
(Fig. \ref{fig:block-diag}). The experimental results obtained within
\proj serve to illustrate and validate the proposed approach,
highlighting the practical benefits and challenges of real-world
adaptive ocean exploration. Furthermore, we demonstrate that this
sampling robotic system offers unprecedented capabilities for ocean
studies by providing samples in close to real-time and enabling
scientists to adaptively re-task AUVs within minutes if some
interesting feature is detected.

