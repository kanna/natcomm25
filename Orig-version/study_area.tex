\section{Study Area}
\label{sec:study-area}

\begin{figure}[!t]
  \vspace{-0.5cm}
  \centering
  \subfigure[Map of Portugal and the study area highlighted with the
  red rectangle.]{\label{fig:po-map}\includegraphics[scale=0.2]{fig/fig_pt_new.png}}
  \hspace{+0.3cm}
  \subfigure[Bathymetry showing the \naz canyon-Berlengas area
  and its environment.]{\label{fig:domain}\includegraphics[scale=0.388]{fig/area_ripples.png}}
  \caption{\subref{fig:po-map} \& \subref{fig:domain} show detailed
    views of the study area for \proj off the coast of mainland
    Portugal in October 2024; the white bands show commercial shipping
    lanes and include two buoys part of a coastal observing system.
    The \naz Canyon is a significant feature of this area and a driver
    for the bio-geophysics of the domain \cite{tyler2009europe}.}
  \label{fig:studyarea-1}
\end{figure}


In this study, we address the challenge of loop closure by
implementing and evaluating a complete data cycle — from model
prediction, to uncertainty generation, to adaptive sampling and data
assimilation. The work was conducted within the framework of \proj
(\textbf{F}ield expe\textbf{R}iments for mod\textbf{E}ling,
a\textbf{S}similatio\textbf{N} and adaptiv\textbf{E}
samp\textbf{L}ing), specifically designed to explore and test
model-driven robotic exploration strategies.

The study was conducted off central Portugal, focusing on the region
influenced by the \naz Canyon (39.2$^{\circ}$ - 39.9$^{\circ}$N)
(Fig. \ref{fig:studyarea-1}) during the Fall of 2024. While the
experiment considered the measurable impact of model-driven robotic
sampling to the bio-geochemistry of the canyon region, this manuscript
is focused on the implications of closing the
\emph{sample-assimilate-predict-direct} loop closure.

Implementing such a closed-loop system in a dynamic coastal
environment presents substantial challenges. These included the need
for spatial high-resolution numerical models capable of rapidly
generating uncertainty predictions, robust algorithms for exploration
under a range of constraints, reliable communication links for mission
updates, and assimilation frameworks that integrate heterogeneous
real-time data streams. Furthermore, marine operations face logistical
risks and communication limitations — such as intermittent
connectivity, unpredictable weather, and vessel traffic, that further
constrain the execution of robotic missions.

The \naz area features pronounced topographic contrasts: the
transition from the wide Estremadura Plateau to the narrower northern
shelf, the long and narrow \naz submarine canyon that incises the
shelf and extends more than 200km offshore, and the Berlengas
archipelago, a UNESCO Biosphere Reserve with high ecological
value. These features contribute to enhanced biological productivity
and biodiversity, and strongly modulate physical and biogeochemical
processes in the region.

Freshwater inputs from major rivers, such as the Tagus, have only a
limited direct impact on the area. In contrast, smaller rivers and the
\'{O}bidos lagoon located south of \naze, can episodically deliver
low-salinity, nutrient-rich plumes to the shelf. Circulation is driven
by seasonal wind forcing linked with the Azores High, with persistent
upwelling-favorable northerly winds in summer and frequent downwelling
episodes in winter under southerly winds. The interplay between canyon
topography, shelf circulation, and atmospheric forcing generates
complex mesoscale dynamics, intensified tidal currents, and internal
wave activity that promote strong vertical mixing and cross-shelf
exchanges \cite{martins10,quaresma07}.

The combination of sharp bathymetric gradients, variable forcing, and
rich physical–biogeochemical interactions makes the \naz Canyon region
an ideal natural laboratory to test sampling strategies and evaluate
how model-based uncertainty projections can guide sampling and
assimilation to improve ocean model predictive skill.


\proj itself is part of a larger concept for multi-domain sensing,
observation and exploration, which couples ocean models, autonomous
robotics, Artificial Intelligence and Machine Learning coupled with a
small satellite constellation. Our end goal is to democratize ocean
model prediction by integrating methods with spatio-temporal data
assimilation from mobile or immobile robots across space, aerial,
surface and underwater domains \cite{rajan21}.
